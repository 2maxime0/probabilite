\documentclass[11pt]{article}

\usepackage[T1]{fontenc}
\usepackage[french]{babel}
\usepackage{amsmath}
\usepackage{amssymb}

\begin{document}

\title{Probabilités et Statistiques}
\author{Maxime FROMONT}
\maketitle

jonathan.bosse@onera.fr

\section{Rappels sur les probabilités et statistiques}

Une variable aléatoire (v.a) est une fonction $X$ associée à une expérience aléatoire, qui associe à un événement $\omega$ de l'univers $\Omega$ une valeur $X(\omega) = x$.
\begin{equation}
\begin{split}
X : \ & \Omega \longrightarrow \mathbb{R} \\
& \omega \longmapsto X(\omega)
\end{split}
\end{equation}

Exemple : lancé de deux pièces de monnaie :
\[Omega = \{PP, PF, FF, FP\}\]
Soit $X$ le nombre de "pile" obtenus
\[X(\Omega) = \{0, 1, 2\}\]

\subsection{Lois de probabilités à valeurs discrètes}

La loi de probabilité d'une v.a est constituée de l'ensemble des couples $(x_i, p_i)$ où \[p_i = P_i(X = x_i)\]

On a $P_i = 1$.

On appelle la fonction de répartition d'une v.a la grandeur 
\[F(x) = Pr(X \leqslant x) = \sum_{x_i \leqslant x} p_i\]

On a si $a < b$, \[Pr(a \leqslant X \leqslant b) = F(b) - F(a)\]

\subsection{Densité de probabilité d'une v.a continue}

Soit $f$ une fonction positive, intégrable et telle que $\int_{-\infty}^{\infty} f(t) dt = 1$, alors $f$ est une fonction de densité.
Une v.a $X$ a pour fonction de densité $f$ si
\[Pr(a \leqslant X \leqslant b) = \int_{a}^{b} f(t) dt\]

La fonction de répartition $F$ est définie par
\[F(x) = Pr(X \leqslant x) = \int_{-\infty}^{x} f(t) dt\]
On a toujours
\[Pr(a \leqslant X \leqslant b) = F(b) - F(a)\]

\subsection{Espérance mathématique \(E(X)\)}

\begin{equation}
\begin{split}
E(X)  & = \sum_{i} x_i p_i \text{(cas discret)} \\
& = \int_{-\infty}^{\infty} t f(t) dt \text{(cas continu)}
\end{split}
\end{equation}

\subsection{Variance d'une v.a \(V(X)\)}

\begin{equation}
\begin{split}
V(X) & = E((X - E(X))^2) = E(X^2) - E(X)^2 \\
& = \sum_{i} x_i^2 p_i - (\sum_{i} x_i p_i)^2 \; \text{(cas discret)} \\
& = \int t^2 f(t) dt - (\int t f(t) dt)^2 \; \text{(cas continu)}
\end{split}
\end{equation}

\section{Lois de probabilités usuelles}

\subsection{Loi de Bernoulli \(B(1, p)\)}

On considère une expérience de type "succès" \((X = 1)\), "échec" \((X = 0)\).
\

\end{document}